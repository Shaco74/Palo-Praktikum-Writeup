\documentclass{article}

\author{Kacper Cömlek}
\title{Palo Praktikum 2 - Writeup}
\date{06.05.2024}

\usepackage{multirow}

\newcommand{\Aufgabe}[2]{\PSN{Aufgabe}{#1}{#2}}

\begin{document}

\maketitle
\tableofcontents 


\section{Aufgabenstellung}

\subsection{Aufgabe 1}
  
\begin{itemize}
	  
	\item Beispielprogramm: 020\_06\_CompilerExercise
	        
	      \begin{itemize}
	      	\item In den Home Ordner kopieren
	      	\item Programm mit verschiedenen Compileroptionen aufrufen und \\Laufzeit notieren
	      \end{itemize}
	      
	\item Das Programm hat mehrere Durchläufe, berechne die mittlere gerundete Laufzeit
	        
	      \begin{itemize}
	      	\item Außreißer wie z.B. der erste Durchlauf werden ignoriert
	      \end{itemize}
	      
	\item Die Ergebnisse haben eine Checksum und Total, die überprüft werden müssen. 
	      Wenn die Optimierung die Checksum verändert, ist das Ergebnis falsch. 
	      In der Tabelle kann dann \emph{n.a. (nicht anwendbar)} eingetragen werden.
	      
	\item Sofern nicht anders angegben verwenden wir Floating-Point Modell precise. 
	      Ggf. explizit einstellen. 
	        
	\item In den Einzelnen Schritten Report Dateien anlegen und versuchen anhand 
	      dessen die Beobachtungen zu erklären. 
\end{itemize}

\subsection{Aufgabe 2}


\section{Aufgabenlösung}

\subsection{Aufgabe 1 Lösung} 
Meine Lösung für Aufgabe 1 ist in Tabelle \ref{tab:aufgabe1} dargestellt\dots

\subsubsection{Aufgabe 1 - Compileroptionen}
Meine Compileroptionen \& Checksum
\begin{verbatim}
#ICC Compilation
CXX_ICC = icpc
CXXFLAGS_ICC = -fopenmp -fp-model=precise -O0
OPTFLAGS_ICC = -qopt-report=5 -qopt-report-file=$@.optrpt
LDFLAGS_ICC =  -qopenmp -qopt-report-file=ipo_test.optrpt 
               -qopt-report=5

Total = 19590.488262 Check Sum = 5001000000

#ICC Compilation für ipo
CXX_ICC = icpc
CXXFLAGS_ICC = -fopenmp -fp-model=precise -O3 -ipo
OPTFLAGS_ICC = -qopt-report=5
LDFLAGS_ICC = -qopenmp -qopt-report-file=ipo_test.optrpt 
              -qopt-report=5
\end{verbatim}

\subsubsection{Aufgabe 1 - Laufzeiten}

\begin{table}[h]
	\centering
	\caption{Optimization Options}
	\label{tab:aufgabe1}
	\begin{tabular}{|c|c|}
		\hline
		\textbf{Optionen} & \textbf{Mittlere Laufzeit} \\
		\hline
		/00 (keine Optimierung) & 4,1786s \\ 
		\hline
		/02 (default Optimierung) & 4,1032s \\
		\hline
		/03 (aggressive Optimierung) & 4,1929s \\
		\hline
		% multicol
		\multicolumn{2}{|c|}{\textbf{Wählen Sie für die folgenden Schritte das Setting /03}} \\
		\hline
		Prozessorspezifische Optimierung -xHost & 4,0935s \\
		\hline
		Interprozeduale Optimierung (multi-File) -ipo & 3,5788s \\
		\hline
		Interprozeduale + Prozessorspezifische Optimierung & 3,5753s \\
		\hline
		%multirow
		\multirow{3}{*}{Prozessorspezifische Optimierung + } &  \\
		  &   \\  Interprozeduale Optimierung (multi-File) & 2,4554s \\
		  &   \\  + Floating Point Model /fast & \\
		\hline
	\end{tabular}
\end{table}

\end{document} 
